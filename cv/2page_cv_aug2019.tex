%%%%%%%%%%%%%%%%%%%%%%%%%%%%%%%%%%%%%%%%%%%%%%%%%%%%%%%%%%%%%%%%%%%%%%%%
  %%%%%%%%%%%%%%%%%%%%%% Simple LaTeX CV Template %%%%%%%%%%%%%%%%%%%%%%%%
  %%%%%%%%%%%%%%%%%%%%%%%%%%%%%%%%%%%%%%%%%%%%%%%%%%%%%%%%%%%%%%%%%%%%%%%%
  
  %%%%%%%%%%%%%%%%%%%%%%%%%%%%%%%%%%%%%%%%%%%%%%%%%%%%%%%%%%%%%%%%%%%%%%%%
  %% NOTE: If you find that it says                                     %%
  %%                                                                    %%
  %%                           1 of ??                                  %%
  %%                                                                    %%
  %% at the bottom of your first page, this means that the AUX file     %%
  %% was not available when you ran LaTeX on this source. Simply RERUN  %%
  %% LaTeX to get the ``??'' replaced with the number of the last page  %%
  %% of the document. The AUX file will be generated on the first run   %%
  %% of LaTeX and used on the second run to fill in all of the          %%
  %% references.                                                        %%
  %%%%%%%%%%%%%%%%%%%%%%%%%%%%%%%%%%%%%%%%%%%%%%%%%%%%%%%%%%%%%%%%%%%%%%%%
  
  %%%%%%%%%%%%%%%%%%%%%%%%%%%% Document Setup %%%%%%%%%%%%%%%%%%%%%%%%%%%%
  
  % Don't like 10pt? Try 11pt or 12pt
\documentclass[11pt]{article}

% The automated optical recognition software used to digitize resume
% information works best with fonts that do not have serifs. This
% command uses a sans serif font throughout. Uncomment both lines (or at
% least the second) to restore a Roman font (i.e., a font with serifs).
%\usepackage{times}
%\renewcommand{\familydefault}{\sfdefault}

% This is a helpful package that puts math inside length specifications
\usepackage{calc}
\usepackage{comment}

% Packages for multi-column reference

% End package for multi-column reference

% Simpler bibsection for CV sections
% (thanks to natbib for inspiration)
\makeatletter
\newlength{\bibhang}
\setlength{\bibhang}{1em} %1em}
\newlength{\bibsep}
{\@listi \global\bibsep\itemsep \global\advance\bibsep by\parsep}
\newenvironment{bibsection}%
{\begin{enumerate}{}{%
%        {\begin{list}{}{%
\setlength{\leftmargin}{\bibhang}%
\setlength{\itemindent}{-\leftmargin}%
\setlength{\itemsep}{\bibsep}%
\setlength{\parsep}{\z@}%
\setlength{\partopsep}{0pt}%
\setlength{\topsep}{0pt}}}
{\end{enumerate}\vspace{-.6\baselineskip}}
%        {\end{list}\vspace{-.6\baselineskip}}
\makeatother

% Layout: Puts the section titles on left side of page
\reversemarginpar

%
%         PAPER SIZE, PAGE NUMBER, AND DOCUMENT LAYOUT NOTES:
%
% The next \usepackage line changes the layout for CV style section
% headings as marginal notes. It also sets up the paper size as either
% letter or A4. By default, letter was used. If A4 paper is desired,
% comment out the letterpaper lines and uncomment the a4paper lines.
%
% As you can see, the margin widths and section title widths can be
% easily adjusted.
%
% ALSO: Notice that the includefoot option can be commented OUT in order
% to put the PAGE NUMBER *IN* the bottom margin. This will make the
% effective text area larger.
%
% IF YOU WISH TO REMOVE THE ``of LASTPAGE'' next to each page number,
% see the note about the +LP and -LP lines below. Comment out the +LP
% and uncomment the -LP.
%
% IF YOU WISH TO REMOVE PAGE NUMBERS, be sure that the includefoot line
% is uncommented and ALSO uncomment the \pagestyle{empty} a few lines
% below.
%

%% Use these lines for letter-sized paper
\usepackage[paper=letterpaper,
%includefoot, % Uncomment to put page number above margin
marginparwidth=1.2in,     % Length of section titles
marginparsep=.05in,       % Space between titles and text
margin=0.75in,             % 1 inch margins
includemp]{geometry}

%% Use these lines for A4-sized paper
%\usepackage[paper=a4paper,
%            %includefoot, % Uncomment to put page number above margin
%            marginparwidth=30.5mm,    % Length of section titles
%            marginparsep=1.5mm,       % Space between titles and text
%            margin=25mm,              % 25mm margins
%            includemp]{geometry}

%% More layout: Get rid of indenting throughout entire document
\setlength{\parindent}{0in}

\usepackage[shortlabels]{enumitem}

%% Reference the last page in the page number
%
% NOTE: comment the +LP line and uncomment the -LP line to have page
%       numbers without the ``of ##'' last page reference)
%
% NOTE: uncomment the \pagestyle{empty} line to get rid of all page
%       numbers (make sure includefoot is commented out above)
%
\usepackage{fancyhdr,lastpage}
\pagestyle{fancy}
%\pagestyle{empty}      % Uncomment this to get rid of page numbers
\fancyhf{}\renewcommand{\headrulewidth}{0pt}
\fancyfootoffset{\marginparsep+\marginparwidth}
\newlength{\footpageshift}
\setlength{\footpageshift}
{0.5\textwidth+0.5\marginparsep+0.5\marginparwidth-1in}
\lfoot{\hspace{\footpageshift}%
\parbox{4in}{\, \hfill %
\arabic{page} of \protect\pageref*{LastPage} % +LP
%                    \arabic{page}                               % -LP
\hfill \,}}

% Finally, give us PDF bookmarks
\usepackage{color,hyperref}
\definecolor{blue}{rgb}{0.0,0.0,1}
\hypersetup{colorlinks,breaklinks,
linkcolor=blue,urlcolor=blue,
anchorcolor=blue,citecolor=blue}

\usepackage{multicol}
%%%%%%%%%%%%%%%%%%%%%%%% End Document Setup %%%%%%%%%%%%%%%%%%%%%%%%%%%%


%%%%%%%%%%%%%%%%%%%%%%%%%%% Helper Commands %%%%%%%%%%%%%%%%%%%%%%%%%%%%

% The title (name) with a horizontal rule under it
% (optional argument typesets an object right-justified across from name
%  as well)
%
% Usage: \makeheading{name}
%        OR
%        \makeheading[right_object]{name}
%
% Place at top of document. It should be the first thing.
% If ``right_object'' is provided in the square-braced optional
% argument, it will be right justified on the same line as ``name'' at
% the top of the CV. For example:
%
%       \makeheading[\emph{Curriculum vitae}]{Your Name}
%
% will put an emphasized ``Curriculum vitae'' at the top of the document
% as a title. Likewise, a picture could be included:
%
%   \makeheading[\includegraphics[height=1.5in]{my_picutre}]{Your Name}
%
% the picture will be flush right across from the name.
\newcommand{\makeheading}[2][]%
{\hspace*{-\marginparsep minus \marginparwidth}%
\begin{minipage}[t]{\textwidth+\marginparwidth+\marginparsep}%
{\large \bfseries #2 \hfill #1}\\[-0.15\baselineskip]%
\rule{\columnwidth}{1pt}%
\end{minipage}}

% The section headings
%
% Usage: \section{section name}
\renewcommand{\section}[1]{\pagebreak[3]%
\hyphenpenalty=10000%
\vspace{1.3\baselineskip}%
\phantomsection\addcontentsline{toc}{section}{#1}%
\noindent\llap{\scshape\smash{\parbox[t]{\marginparwidth}{\raggedright #1}}}%
\vspace{-\baselineskip}\par}

% An itemize-style list with lots of space between items
\newenvironment{outerlist}[1][\enskip\textbullet]%
{\begin{itemize}[#1,leftmargin=*]}{\end{itemize}%
\vspace{-.6\baselineskip}}

% An environment IDENTICAL to outerlist that has better pre-list spacing
% when used as the first thing in a \section
\newenvironment{lonelist}[1][\enskip\textbullet]%
{\begin{list}{#1}{%
\setlength{\partopsep}{0pt}%
\setlength{\topsep}{0pt}}}
{\end{list}\vspace{-.6\baselineskip}}

% An itemize-style list with little space between items
\newenvironment{innerlist}[1][\enskip\textbullet]%
{\begin{itemize}[#1,leftmargin=*,parsep=0pt,itemsep=0pt,topsep=0pt,partopsep=0pt]}
{\end{itemize}}

% An environment IDENTICAL to innerlist that has better pre-list spacing
% when used as the first thing in a \section
\newenvironment{loneinnerlist}[1][\enskip\textbullet]%
{\begin{itemize}[#1,leftmargin=*,parsep=0pt,itemsep=0pt,topsep=0pt,partopsep=0pt]}
{\end{itemize}\vspace{-.6\baselineskip}}

% To add some paragraph space between lines.
% This also tells LaTeX to preferably break a page on one of these gaps
% if there is a needed pagebreak nearby.
\newcommand{\blankline}{\quad\pagebreak[3]}
\newcommand{\halfblankline}{\quad\vspace{-0.5\baselineskip}\pagebreak[3]}

% Uses hyperref to link DOI
\newcommand\doilink[1]{\href{http://dx.doi.org/#1}{#1}}
\newcommand\doi[1]{doi:\doilink{#1}}

% For \url{SOME_URL}, links SOME_URL to the url SOME_URL
\providecommand*\url[1]{\href{#1}{#1}}
% Same as above, but pretty-prints SOME_URL in teletype fixed-width font
\renewcommand*\url[1]{\href{#1}{\texttt{#1}}}

% For \email{ADDRESS}, links ADDRESS to the url mailto:ADDRESS
\providecommand*\email[1]{\href{mailto:#1}{#1}}
% Same as above, but pretty-prints ADDRESS in teletype fixed-width font
%\renewcommand*\email[1]{\href{mailto:#1}{\texttt{#1}}}

%\providecommand\BibTeX{{\rm B\kern-.05em{\sc i\kern-.025em b}\kern-.08em
%    T\kern-.1667em\lower.7ex\hbox{E}\kern-.125emX}}
%\providecommand\BibTeX{{\rm B\kern-.05em{\sc i\kern-.025em b}\kern-.08em
%    \TeX}}
\providecommand\BibTeX{{B\kern-.05em{\sc i\kern-.025em b}\kern-.08em
\TeX}}
\providecommand\Matlab{\textsc{Matlab}}



%%%%%%%%%%%%%%%%%%%%%%%% End Helper Commands %%%%%%%%%%%%%%%%%%%%%%%%%%%

%%%%%%%%%%%%%%%%%%%%%%%%% Begin CV Document %%%%%%%%%%%%%%%%%%%%%%%%%%%%


\begin{document}
\makeheading{Aniruddha Ghosh}

\section{Contact Information}

% NOTE: Mind where the & separators and \\ breaks are in the following
%       table.
%
% ALSO: \rcollength is the width of the right column of the table
%       (adjust it to your liking; default is 1.85in).
%
\newlength{\rcollength}\setlength{\rcollength}{1.4in}%
%
\begin{tabular}[t]{@{}p{\textwidth-\rcollength}p{\rcollength}}
%\href{http://www.cse.osu.edu/}%
%     {Department of Computer Science and Engineering} & \\
%\href{http://www.osu.edu/}{The Ohio State University}
2009 Wickson Hall & \ (505) 920-1611 \\
DAVIS, CA  95616 & \email{anighosh@ucdavis.edu}\\
\end{tabular}

%\section{Objective}

%Insert text here if you want to
%\begin{innerlist}
%\item More information and auxiliary documents can be found at\\\url{http://www.tedpavlic.com/facjobsearch/}
%\end{innerlist}

\section{Summary}
Spatial data scientist with extensive experience in solving real world problems within \newline interdisciplinary research teams. Passionate about teaching R and developing web-based training materials. Special expertise in the following areas:
Agriculture, Collaborative Research, Machine Learning, Spatial Data Analysis, Remote Sensing

\section{Research Experience}

\textbf{University of California} \hfill {Davis, CA}\\
Assistant Project Scientist; PI: Prof. Robert Hijmans \hfill{July 2015 -- present}
\vspace{.05in}
\begin{innerlist}
        \item Design remote sensing based tools to support agriculture index insurance program
        \item Develop big data and machine learning models for crop monitoring and yield prediction
        \item Develop R-packages and Google Earth Engine based web-apps
        \item Collect, curate and manage wide range of data from secondary sources
        \item Grant management and coordinate collaborative research efforts of \href{gfc.ucdavis.edu}{Geospatial \newline and Farming Systems Research Consortium (GFC)}
        \item Organize workshops (international and national locations) on data science for \newline agriculture with free and open source tools
        \item Support collaborative research projects across multiple research clusters
\end{innerlist}

\vspace{.1in}

\textbf{Colorado State University} \hfill {Fort Collins, CO}\\
Postdoctoral Researcher; Advisor: Prof. Michael Lefsky \hfill {May 2014 to July 2015}
\vspace{.05in}
\begin{innerlist}
        \item Calibrated atmospheric correction methods for data imaging spectrometer instruments
        \item Developed linear unmixing models to study variability within burn severity classes 
\end{innerlist}

\vspace{.1in}

\textbf{TERI University} \hfill{New Delhi, India}\\
Doctoral Researcher; Advisor: Prof. PK Joshi \hfill{Aug 2009 to Feb 2014}
\vspace{.05in}
\begin{innerlist}
        \item Compared predictive models and sensor performance for tree species and bark beetle infestation detection using hyperspectral and LiDAR data
        \item Developed a novel method for land-use/land-cover mapping using time series multi-spectral remote sensing data
        \item Developed image segmention methods for mapping tree species using very high resolution satellite image
        \item Developed data-driven models to improve the spatial resolution of multi-spectral and thermal remote sensing data 
\end{innerlist}

\section{Education}

{\textbf{TERI University}},New Delhi, India \hfill{February 2014}
\newline PhD, Natural Resources Management 

\vspace{.1in}
{\textbf{University of Delhi}},	New Delhi, India  \hfill{July 2009}
\newline MSc, Department of Physics \& Astrophysics

\vspace{.1in}
{\textbf{University of Calcutta}}, Kolkata, India  \hfill{June 2006}
\newline BSc, Physics


\section{Technical Skills
[\href{https://github.com/ani-ghosh}{github}]
}
\begin{innerlist}
    \item Programming Languages: R, Python, JavaScript
    \item Geospatial: Google Earth Engine APIs, ENVI-IDL, ERDAS-Imagine, eCognition, \newline ArcGIS, Pix4D, QGIS
\end{innerlist}

\vspace{.1in} 
Software Development (R-packages)
\begin{innerlist}
    \item \href{https://cran.r-project.org/web/packages/raster/index.html}{raster : Geographic Data Analysis and Modeling} (contributor) 
    \item \href{https://github.com/rspatial/luna}{luna : satellite remote sensing tools} (co-author)
\end{innerlist}

\section{Workshops}
Training workshops (10+) with more than more than 200 participants from 20+ countries

\section{Web-resources}
\begin{innerlist}
\item \href{https://rspatial.org/}{Spatial Data Science with R} (co-author)
\item \href{https://gfc.ucdavis.edu/events/agrin/html/}{Remote sensing for agricultural insurance} (co-author)
\end{innerlist}

\section{Professional Services}
\begin{innerlist}
\item Co-coordinator, Center for Spatial Sciences, University of California Davis
\item Reviewer for 20+ journals
\item Guest Editor, Challenges and Opportunities in Remote Sensing based Agricultural\newline Monitoring in Smallholder Agriculture. \emph{Sensors, 2020}
\item Scientific Committee member of the Academic Track of FOSS4G 2018 (Free and Open\newline Source Software For Geospatial 2018), Tanzania, August 2018
\end{innerlist}

\section{Selected Journal Publications
[\href{https://scholar.google.com/citations?user=jFhtI_AAAAAJ&hl=en}{Google Scholar};
\href{https://www.researchgate.net/profile/Aniruddha_Ghosh5}{Researchgate}]
}
\vspace{-.13in}
\begin{bibsection}
    \item Smith, J., {\bf Ghosh, A.}, Hijmans, R., Agricultural intensification is associated with crop diversification in India. \emph{In review, PLoS ONE.}
    \item Shew, A., {\bf Ghosh, A.}, Identifying Dry Season Rice Planting Patterns in Bangladesh using the Landsat Archive. \emph{Remote Sensing}, 11(10), 2019.
    \item {\bf Ghosh, A.}, Joshi, PK., Hyperspectral imagery for disaggregation of land surface temperature with selected regression algorithms over different land use land cover scenes. \emph{ISPRS Journal of Photogrammetry and Remote Sensing}, 96, 76--93, 2014.
    \item {\bf Ghosh, A.}, Sharma, R., Joshi, PK., Random forest classification of urban landscape using Landsat archive and ancillary data: Combining seasonal maps with decision level fusion. \emph{Applied Geography}, 48, 31--41, 2014.
    \item Fassnacht, FE., Latifi, H., {\bf Ghosh, A.}, Joshi, PK., Koch, B., Assessing the potential of hyperspectral imagery to map bark beetle-induced tree mortality. \emph{Remote Sensing of Environment}, 140, 533--548, 2014.

\end{bibsection}

\section{Selected Conference Presentations}
\vspace{-.1275in}
\begin{bibsection}
\item Wang, H., {\bf Ghosh, A.}, Mapping paddy rice cropping pattern and phenology in Cambodia. \emph{FOSS4G 2018, Dar es Salaam, Tanzania.} \hfill{Aug 2018}
\item Tiedeman, K., {\bf Ghosh, A.}, Landcover classification with R and Google Earth Engine to predict Human-elephant conflict. \emph{FOSS4G 2018, Dar es Salaam, \\Tanzania.} \hfill{Aug 2018} 
\item {\bf Ghosh, A.}, Mandel, A., Hijmans, R., Developing Scalable Information Extraction Processing Pipelines using R for Earth Observation Applications.\emph{FOSS4G 2017, Boston, USA.} \hfill{Aug 2017}
\item Mandel, A., {\bf Ghosh, A.}, Hijmans, R., Rspatial.org, tutorials for learning Spatial R.\emph{FOSS4G 2017, Boston, USA.} \hfill{Aug 2017}
\item {\bf Ghosh, A.}, Joshi, PK., Identification of bamboo patches in the lower Gangetic plains using very high resolution WorldView 2 imagery. \emph{SPIE Remote Sensing, Dresden, Germany.}\hfill Sep 2013

\end{bibsection}

\end{document}

%%%%%%%%%%%%%%%%%%%%%%%%%% End CV Document %%%%%%%%%%%%%%%%%%%%%%%%%%%%%

%----------------------------------------------------------------------%
  % The following is copyright and licensing information for
% redistribution of this LaTeX source code; it also includes a liability
% statement. If this source code is not being redistributed to others,
% it may be omitted. It has no effect on the function of the above code.
%----------------------------------------------------------------------%
  % Copyright (c) 2007, 2008, 2009, 2010, 2011 by Theodore P. Pavlic
%
% Unless otherwise expressly stated, this work is licensed under the
% Creative Commons Attribution-Noncommercial 3.0 United States License. To
% view a copy of this license, visit
% http://creativecommons.org/licenses/by-nc/3.0/us/ or send a letter to
% Creative Commons, 171 Second Street, Suite 300, San Francisco,
% California, 94105, USA.
%
% THE SOFTWARE IS PROVIDED "AS IS", WITHOUT WARRANTY OF ANY KIND, EXPRESS
% OR IMPLIED, INCLUDING BUT NOT LIMITED TO THE WARRANTIES OF
% MERCHANTABILITY, FITNESS FOR A PARTICULAR PURPOSE AND NONINFRINGEMENT.
% IN NO EVENT SHALL THE AUTHORS OR COPYRIGHT HOLDERS BE LIABLE FOR ANY
% CLAIM, DAMAGES OR OTHER LIABILITY, WHETHER IN AN ACTION OF CONTRACT,
% TORT OR OTHERWISE, ARISING FROM, OUT OF OR IN CONNECTION WITH THE
% SOFTWARE OR THE USE OR OTHER DEALINGS IN THE SOFTWARE.
%----------------------------------------------------------------------%
  
  