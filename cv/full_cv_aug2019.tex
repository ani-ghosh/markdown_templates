%%%%%%%%%%%%%%%%%%%%%%%%%%%%%%%%%%%%%%%%%%%%%%%%%%%%%%%%%%%%%%%%%%%%%%%%
%%%%%%%%%%%%%%%%%%%%%% Simple LaTeX CV Template %%%%%%%%%%%%%%%%%%%%%%%%
%%%%%%%%%%%%%%%%%%%%%%%%%%%%%%%%%%%%%%%%%%%%%%%%%%%%%%%%%%%%%%%%%%%%%%%%

%%%%%%%%%%%%%%%%%%%%%%%%%%%%%%%%%%%%%%%%%%%%%%%%%%%%%%%%%%%%%%%%%%%%%%%%
%% NOTE: If you find that it says                                     %%
%%                                                                    %%
%%                           1 of ??                                  %%
%%                                                                    %%
%% at the bottom of your first page, this means that the AUX file     %%
%% was not available when you ran LaTeX on this source. Simply RERUN  %%
%% LaTeX to get the ``??'' replaced with the number of the last page  %%
%% of the document. The AUX file will be generated on the first run   %%
%% of LaTeX and used on the second run to fill in all of the          %%
%% references.                                                        %%
%%%%%%%%%%%%%%%%%%%%%%%%%%%%%%%%%%%%%%%%%%%%%%%%%%%%%%%%%%%%%%%%%%%%%%%%

%%%%%%%%%%%%%%%%%%%%%%%%%%%% Document Setup %%%%%%%%%%%%%%%%%%%%%%%%%%%%

% Don't like 10pt? Try 11pt or 12pt
\documentclass[11pt]{article}

% The automated optical recognition software used to digitize resume
% information works best with fonts that do not have serifs. This
% command uses a sans serif font throughout. Uncomment both lines (or at
% least the second) to restore a Roman font (i.e., a font with serifs).
%\usepackage{times}
%\renewcommand{\familydefault}{\sfdefault}

% This is a helpful package that puts math inside length specifications
\usepackage{calc}
\usepackage{comment}

% Packages for multi-column reference

% End package for multi-column reference

% Simpler bibsection for CV sections
% (thanks to natbib for inspiration)
\makeatletter
\newlength{\bibhang}
\setlength{\bibhang}{1em} %1em}
\newlength{\bibsep}
 {\@listi \global\bibsep\itemsep \global\advance\bibsep by\parsep}
\newenvironment{bibsection}%
        {\begin{enumerate}{}{%
%        {\begin{list}{}{%
       \setlength{\leftmargin}{\bibhang}%
       \setlength{\itemindent}{-\leftmargin}%
       \setlength{\itemsep}{\bibsep}%
       \setlength{\parsep}{\z@}%
        \setlength{\partopsep}{0pt}%
        \setlength{\topsep}{0pt}}}
        {\end{enumerate}\vspace{-.6\baselineskip}}
%        {\end{list}\vspace{-.6\baselineskip}}
\makeatother

% Layout: Puts the section titles on left side of page
\reversemarginpar

%
%         PAPER SIZE, PAGE NUMBER, AND DOCUMENT LAYOUT NOTES:
%
% The next \usepackage line changes the layout for CV style section
% headings as marginal notes. It also sets up the paper size as either
% letter or A4. By default, letter was used. If A4 paper is desired,
% comment out the letterpaper lines and uncomment the a4paper lines.
%
% As you can see, the margin widths and section title widths can be
% easily adjusted.
%
% ALSO: Notice that the includefoot option can be commented OUT in order
% to put the PAGE NUMBER *IN* the bottom margin. This will make the
% effective text area larger.
%
% IF YOU WISH TO REMOVE THE ``of LASTPAGE'' next to each page number,
% see the note about the +LP and -LP lines below. Comment out the +LP
% and uncomment the -LP.
%
% IF YOU WISH TO REMOVE PAGE NUMBERS, be sure that the includefoot line
% is uncommented and ALSO uncomment the \pagestyle{empty} a few lines
% below.
%

%% Use these lines for letter-sized paper
\usepackage[paper=letterpaper,
            %includefoot, % Uncomment to put page number above margin
            marginparwidth=1.2in,     % Length of section titles
            marginparsep=.05in,       % Space between titles and text
            margin=0.7in,               % 1 inch margins
            includemp]{geometry}

%% Use these lines for A4-sized paper
%\usepackage[paper=a4paper,
%            %includefoot, % Uncomment to put page number above margin
%            marginparwidth=30.5mm,    % Length of section titles
%            marginparsep=1.5mm,       % Space between titles and text
%            margin=25mm,              % 25mm margins
%            includemp]{geometry}

%% More layout: Get rid of indenting throughout entire document
\setlength{\parindent}{0in}

\usepackage[shortlabels]{enumitem}

%% Reference the last page in the page number
%
% NOTE: comment the +LP line and uncomment the -LP line to have page
%       numbers without the ``of ##'' last page reference)
%
% NOTE: uncomment the \pagestyle{empty} line to get rid of all page
%       numbers (make sure includefoot is commented out above)
%
\usepackage{fancyhdr,lastpage}
\pagestyle{fancy}
%\pagestyle{empty}      % Uncomment this to get rid of page numbers
\fancyhf{}\renewcommand{\headrulewidth}{0pt}
\fancyfootoffset{\marginparsep+\marginparwidth}
\newlength{\footpageshift}
\setlength{\footpageshift}
          {0.5\textwidth+0.5\marginparsep+0.5\marginparwidth-1in}
\lfoot{\hspace{\footpageshift}%
       \parbox{4in}{\, \hfill %
                    \arabic{page} of \protect\pageref*{LastPage} % +LP
%                    \arabic{page}                               % -LP
                    \hfill \,}}

% Finally, give us PDF bookmarks
\usepackage{color,hyperref}
\definecolor{blue}{rgb}{0.0,0.0,1}
\hypersetup{colorlinks,breaklinks,
            linkcolor=blue,urlcolor=blue,
            anchorcolor=blue,citecolor=blue}

\usepackage{multicol}
%%%%%%%%%%%%%%%%%%%%%%%% End Document Setup %%%%%%%%%%%%%%%%%%%%%%%%%%%%


%%%%%%%%%%%%%%%%%%%%%%%%%%% Helper Commands %%%%%%%%%%%%%%%%%%%%%%%%%%%%

% The title (name) with a horizontal rule under it
% (optional argument typesets an object right-justified across from name
%  as well)
%
% Usage: \makeheading{name}
%        OR
%        \makeheading[right_object]{name}
%
% Place at top of document. It should be the first thing.
% If ``right_object'' is provided in the square-braced optional
% argument, it will be right justified on the same line as ``name'' at
% the top of the CV. For example:
%
%       \makeheading[\emph{Curriculum vitae}]{Your Name}
%
% will put an emphasized ``Curriculum vitae'' at the top of the document
% as a title. Likewise, a picture could be included:
%
%   \makeheading[\includegraphics[height=1.5in]{my_picutre}]{Your Name}
%
% the picture will be flush right across from the name.
\newcommand{\makeheading}[2][]%
        {\hspace*{-\marginparsep minus \marginparwidth}%
         \begin{minipage}[t]{\textwidth+\marginparwidth+\marginparsep}%
             {\large \bfseries #2 \hfill #1}\\[-0.15\baselineskip]%
                 \rule{\columnwidth}{1pt}%
         \end{minipage}}

% The section headings
%
% Usage: \section{section name}
\renewcommand{\section}[1]{\pagebreak[3]%
    \hyphenpenalty=10000%
    \vspace{1.3\baselineskip}%
    \phantomsection\addcontentsline{toc}{section}{#1}%
    \noindent\llap{\scshape\smash{\parbox[t]{\marginparwidth}{\raggedright #1}}}%
    \vspace{-\baselineskip}\par}

% An itemize-style list with lots of space between items
\newenvironment{outerlist}[1][\enskip\textbullet]%
        {\begin{itemize}[#1,leftmargin=*]}{\end{itemize}%
         \vspace{-.6\baselineskip}}

% An environment IDENTICAL to outerlist that has better pre-list spacing
% when used as the first thing in a \section
\newenvironment{lonelist}[1][\enskip\textbullet]%
        {\begin{list}{#1}{%
        \setlength{\partopsep}{0pt}%
        \setlength{\topsep}{0pt}}}
        {\end{list}\vspace{-.6\baselineskip}}

% An itemize-style list with little space between items
\newenvironment{innerlist}[1][\enskip\textbullet]%
        {\begin{itemize}[#1,leftmargin=*,parsep=0pt,itemsep=0pt,topsep=0pt,partopsep=0pt]}
        {\end{itemize}}

% An environment IDENTICAL to innerlist that has better pre-list spacing
% when used as the first thing in a \section
\newenvironment{loneinnerlist}[1][\enskip\textbullet]%
        {\begin{itemize}[#1,leftmargin=*,parsep=0pt,itemsep=0pt,topsep=0pt,partopsep=0pt]}
        {\end{itemize}\vspace{-.6\baselineskip}}

% To add some paragraph space between lines.
% This also tells LaTeX to preferably break a page on one of these gaps
% if there is a needed pagebreak nearby.
\newcommand{\blankline}{\quad\pagebreak[3]}
\newcommand{\halfblankline}{\quad\vspace{-0.5\baselineskip}\pagebreak[3]}

% Uses hyperref to link DOI
\newcommand\doilink[1]{\href{http://dx.doi.org/#1}{#1}}
\newcommand\doi[1]{doi:\doilink{#1}}

% For \url{SOME_URL}, links SOME_URL to the url SOME_URL
\providecommand*\url[1]{\href{#1}{#1}}
% Same as above, but pretty-prints SOME_URL in teletype fixed-width font
\renewcommand*\url[1]{\href{#1}{\texttt{#1}}}

% For \email{ADDRESS}, links ADDRESS to the url mailto:ADDRESS
\providecommand*\email[1]{\href{mailto:#1}{#1}}
% Same as above, but pretty-prints ADDRESS in teletype fixed-width font
%\renewcommand*\email[1]{\href{mailto:#1}{\texttt{#1}}}

%\providecommand\BibTeX{{\rm B\kern-.05em{\sc i\kern-.025em b}\kern-.08em
%    T\kern-.1667em\lower.7ex\hbox{E}\kern-.125emX}}
%\providecommand\BibTeX{{\rm B\kern-.05em{\sc i\kern-.025em b}\kern-.08em
%    \TeX}}
\providecommand\BibTeX{{B\kern-.05em{\sc i\kern-.025em b}\kern-.08em
    \TeX}}
\providecommand\Matlab{\textsc{Matlab}}



%%%%%%%%%%%%%%%%%%%%%%%% End Helper Commands %%%%%%%%%%%%%%%%%%%%%%%%%%%

%%%%%%%%%%%%%%%%%%%%%%%%% Begin CV Document %%%%%%%%%%%%%%%%%%%%%%%%%%%%

\begin{document}
\makeheading{Aniruddha Ghosh}

\section{Contact Information}

% NOTE: Mind where the & separators and \\ breaks are in the following
%       table.
%
% ALSO: \rcollength is the width of the right column of the table
%       (adjust it to your liking; default is 1.85in).
%
\newlength{\rcollength}\setlength{\rcollength}{1.4in}%
%
\begin{tabular}[t]{@{}p{\textwidth-\rcollength}p{\rcollength}}
Alliance of Bioversity International and CIAT, Africa Hub & \ +254(796)060763 \\
Duduville Campus Off Kasarani Road, Nairobi, Kenya & \email{A.Ghosh@cgiar.org}\\
\end{tabular}

%\section{Objective}

%Insert text here if you want to
%\begin{innerlist}
%\item More information and auxiliary documents can be found at\\\url{http://www.tedpavlic.com/facjobsearch/}
%\end{innerlist}

\section{Academic Appointments}

\textbf{University of California, Davis} \hfill {Davis, CA}\\
Assistant Project Scientist; PI: Prof. Robert Hijmans \hfill{July 2015 -- present}

\begin{outerlist}
\item[] Design and coordinate research projects, develop spatial data analytics techniques and support outreach activities for agricultural development and conservation projects
    \begin{innerlist}
        \item Design remote sensing based tools to support agriculture index insurance program
        \item Develop big data and machine learning analytics tools for agriculture monitoring and crop yield prediction
        \item Design crop monitoring workflow with combination of various in-situ sensors and unmanned aerial vehicle (UAV)
        \item Develop remote sensing based monitoring frameworks for conservation applications
        \item Generate and distribute wide range of geospatial variables for agricultural research
        \item Grant management and coordinate collaborative research efforts of the \href{gfc.ucdavis.edu}{Geospatial and Farming Systems Research Consortium (GFC)}
        \item Organize workshops (international and national locations) on spatial data science for agriculture with free and open source tools
    \end{innerlist}
\end{outerlist}
\vspace{.12in}

\textbf{Colorado State University} \hfill {Fort Collins, CO}\\
Postdoctoral Researcher; Advisor: Prof. Michael Lefsky \hfill {May 2014 to July 2015}

\begin{outerlist}
\item[]Characterize post-fire vegetation recovery using airborne hyperspectral and LiDAR data collected by NEON Airborne Observation Platform (AOP)
    \begin{innerlist}
        \item Developed atmospheric correction methods for data collected by NEON imaging spectrometer instruments 
        \item Studied variability within burn severity classes using hyperspectral data 
        \item Conducted extensive field survey to collect spectroradiometer measurements of various burnt surface features
        \item Generated pre- and post-fire vegetation type map to understand the effect of wildfire on different vegetation type
    \end{innerlist}
\end{outerlist}
\vspace{.12in}

\textbf{TERI University} \hfill{New Delhi, India}\\
Doctoral Researcher; Advisor: Prof. PK Joshi \hfill{Aug 2009 to Feb 2014}

\begin{outerlist}
\item[]Multi-sensor image fusion for different environmental applications
    \begin{innerlist}
        \item Compared classifier and sensor performance for mapping tree species using\\ hyperspectral and LiDAR data
        \item Developed a novel method for land-use/land-cover mapping with specific focus on agriculture land use practices using time series multi-spectral remote sensing data
        \item Developed methods for mapping tree species mapping in heterogeneous landscapes using very high resolution satellite image
        \item Compared different models to improve the spatial resolution of multi-spectral and thermal remote sensing data 
    \end{innerlist}
\end{outerlist}
\vspace{.1in}

\textbf{University of Freiburg} \hfill{Freiburg, Germany}\\
Visiting Researcher; Mentor: Prof. Barbara Koch \hfill {Sep 2012 to Nov 2012}

\begin{outerlist}
\item[]Hyperspecral and LiDAR data analysis for forestry applications
    \begin{innerlist}
        \item Designed a framework to understand the effect of scale on tree species classification using hyperspectral and LiDAR data
        \item Developed a method to detect bark beetle infestation stages using airborne\\ hyperspectral imagery
    \end{innerlist}
\end{outerlist}


\section{Education}

{\textbf{TERI University}},New Delhi, India
\begin{outerlist}

\item[] Ph.D., Natural Resources Management \hfill{February 2014}
        \begin{innerlist}
        \item Thesis Topic: \emph{Framework	to Unify Sensor Information	for Observing Nature \\(FUSION): Selected Earth observation applications using remote sensing data}
        \item Advisor: Prof. PK Joshi
        \end{innerlist}
\end{outerlist}

\vspace{.1in}
{\textbf{University of Delhi}},	
New Delhi, India
\begin{outerlist}
\item[] M.Sc., Department of Physics \& Astrophysics \hfill{July 2009}
\end{outerlist}

\vspace{.1in}
{\textbf{University of Calcutta}}, Kolkata, India
\begin{outerlist}
\item[] B.Sc., Physics, Presidency College \hfill{June 2006}

\end{outerlist}

\section{Refereed Journal Publications
[\href{https://scholar.google.com/citations?user=jFhtI_AAAAAJ&hl=en}{Google Scholar};
\href{https://www.researchgate.net/profile/Aniruddha_Ghosh5}{Researchgate}]
}

\vspace{-.1275in}
\begin{bibsection}
    \item Smith, J., {\bf Ghosh, A.}, Hijmans, RJ., Agricultural intensification is associated with crop diversification in India. \emph{PLoS ONE}, 14(12), 2019.
    \item Marshall, M., Crommelinck, S., Kohli, D., Perger, C., Yang, M., {\bf Ghosh, A.}, Fritz, S., de Bie, K., Nelson, A. Crowd-driven and automated mapping of field boundaries in highly fragmented agricultural landscapes of Ethiopia with very high spatial resolution imagery. \emph{Remote Sensing}, 11(18), 2019.
    \item Shew, A., {\bf Ghosh, A.}, Identifying Dry Season Rice Planting Patterns in Bangladesh using the Landsat Archive. \emph{Remote Sensing}, 11(10), 2019.
    \item Shew, A., Durand-Morat, A., Putman, B., Nalley, L.L., {\bf Ghosh, A.}, Rice Intensification in Bangladesh Improves Economic and Environmental Welfare. \emph{Environmental Science and Policy}, 95, 46--57, 2019.
    \item Chakraborty, A., {\bf Ghosh, A.}, Sachdeva, K., Joshi, PK., Characterizing \\fragmentation trends of the Himalayan forests in the Kumaon region of Uttarakhand, India. \emph{Ecological Informatics}, 38, 95--109, 2016.
    \item Fassnacht, FE., Latifi, H., Stere{\'n}czak, K., Modzelewska, A., Lefsky, M., Waser, LT., Straub, C., {\bf Ghosh, A.}, Review of studies on tree species classification from remotely sensed data. \emph{Remote Sensing of Environment}, 186(1), 64--87, 2016.
    \item {\bf Ghosh, A.}, Joshi, PK., Hyperspectral imagery for disaggregation of land surface temperature with selected regression algorithms over different land use land cover scenes. \emph{ISPRS Journal of Photogrammetry and Remote Sensing}, 96, 76--93, 2014.
    \item Fassnacht, FE., Neumann, C., Forster, M., Buddenbaum, H., {\bf Ghosh, A.}, Clasen, A., Joshi, PK., Koch, B., Comparison of Feature Reduction Algorithms for\\ Classifying Tree Species With Hyperspectral Data on Three Central European Test Sites. \emph{IEEE Journal of Selected Topics in Applied Earth Observations and Remote Sensing}, 7(6), 2547--2561, 2014.
    \item {\bf Ghosh, A.}, Sharma, R., Joshi, PK., Random forest classification of urban landscape using Landsat archive and ancillary data: Combining seasonal maps with decision level fusion. \emph{Applied Geography}, 48, 31--41, 2014.
    \item Fassnacht, FE., Latifi, H., {\bf Ghosh, A.}, Joshi, PK., Koch, B., Assessing the potential of hyperspectral imagery to map bark beetle-induced tree mortality. \emph{Remote Sensing of Environment}, 140, 533--548, 2014.
    \item {\bf Ghosh, A.}, Joshi, PK., A comparison of selected classification algorithms for mapping bamboo patches in lower Gangetic plains using very high resolution\\ WorldView 2 imagery. \emph {International Journal of Applied Earth Observation and Geoinformation}, 26, 298--311, 2014.
    \item {\bf Ghosh, A.}, Fassnacht, FE., Joshi, PK., Koch, B., ``A framework for mapping tree species combining hyperspectral and LiDAR data: Role of selected classifiers and sensor across three spatial scales." \emph {International Journal of Applied Earth Observation and Geoinformation}, 26, 49--63, 2014.
    \item {\bf Ghosh, A.}, Joshi, PK., Assessment of pan-sharpened very high-resolution\\ WorldView-2 images. \emph{International Journal of Remote Sensing}, 34(23), 8336--8359, 2013.
    \item Chakraborty, A., Joshi, PK., {\bf Ghosh, A.}, Areendran, G., Assessing biome boundary shifts under climate change scenarios in India. \emph{Ecological Indicators}, 34, 536--547, 2013.
    \item Joshi, PK., {\bf Ghosh, A.}, Chakraborty, A., Sharma, R., Joshi, A., Landsat again – continuing remote sensing, monitoring, mapping and measuring. \emph{Current Science}, 105(6), 761--763, 2013.
    \item Biswal, S., {\bf Ghosh, A.}, Sharma, R., Joshi, PK., Satellite Data Classification Using Open Source Support. \emph{Journal of the Indian Society of Remote Sensing}, 41(3), 523--530, 2013.
    \item Sharma, R., {\bf Ghosh, A.}, Joshi, PK., Analysing spatio-temporal footprints of \\urbanization on environment of Surat city using satellite-derived bio-physical\\ parameters. \emph{Geocarto International}, 28(5), 420--438, 2013.
    \item {\bf Ghosh, A.}, Joshi, PK., Remote sensing of moving objects. \emph{Current Science}, 104(12), 1613--1615, 2013.
    \item {\bf Ghosh, A.}, Joshi, PK., Garg, RD., Mukherjee, S., Identification of invasive plant species using field spectro-radiometer and simulated Hyperion spectra -- a rapid mapping of invasiveness. \emph{Current Science}, 104(9), 1148--1151, 2013.
    \item Mukherjee, S., Joshi, PK., Mukherjee, S., {\bf Ghosh, A.}, Garg, RD., Mukhopadhyay, A., Evaluation of vertical accuracy of open source Digital Elevation Model (DEM). \emph {International Journal of Applied Earth Observation and Geoinformation}, 21, 205--207, 2013.
    \item Sharma, R., {\bf Ghosh, A.}, Joshi, PK., Decision tree approach for classification of remotely sensed satellite data using open source support. \emph{Journal of Earth System Science}, 122(5), 1237--1247, 2013.
    \item Sharma, R., {\bf Ghosh, A.}, Joshi, PK., Spatio-temporal footprints of urbanisation in Surat, the Diamond City of India (1990--2009). \emph {Environmental monitoring and assessment}, 185(4), 3313--3325, 2013.
    \item {\bf Ghosh, A.}, Munshi, M., Areendran, G., Joshi, PK., Pattern space analysis of landscape metrics for detecting changes in forests of Himalayan foothills. \emph{Asian Journal of Geoinformatics}, 12(1), 2012.
    \item Joshi, PK., Narula, S., Rawat, A., {\bf Ghosh, A.}, Landscape characterization of Sariska National Park (India) and its surroundings. \emph{Geo-spatial Information Science}, 14(4): 303--310, 2011.
    \item Munsi, M., Areendran, G., {\bf Ghosh, A.}, Joshi, PK.  Landscape characterisation of the forests of Himalayan foothills. \emph{Journal of the Indian Society of Remote Sensing}, 38(3):441--31452, 2010.

\end{bibsection}

\section{Manuscripts In preparation}
\vspace{-.1275in}
\begin{bibsection}
\item Wang, H., {\bf Ghosh, A.}, Linquist, BA., Hijmans, RJ., Satellite-based Observations Reveal Effects of Weather Variation on Rice Phenology in California.
\item Springborn, M.R., Weill, J.A., Lips, K.R., Ibanez, R.,{\bf Ghosh, A.}, Ecosystem Service Impacts of Ecological Disruptions: Evidence From Malaria Incidence in Central America.
\item Tiedeman, K., {\bf Ghosh, A.}, Paul, L., Dougherty, J., Flatnes, JE., Hijmans, RJ. Comparison of MODIS, Landsat 8 and Sentinel for mapping cropland extent and areas under maize in Tanzania.
\item {\bf Ghosh, A.}, Wang, H.,  Hijmans, RJ., Changing paddy rice cropping practices in Cambodia with time series satellite data.
\item Lwehabura, J., Stewart, ZP., Rubyogo, JC., Prasad, PVV., Mason, N., Snapp, S., Kilango, M., {\bf Ghosh, A.}, Geospatial analysis for scaling bean technology adoption in Tanzania.
\item Benami, E., Jin, Z., Kenduiywo, B., {\bf Ghosh, A.}, Carter, M., Lobell, D., Hijmans, RJ., Uniting Advances in Remote Sensing, Crop Modeling, \& Economics to Improve Management of Weather Risk in Agriculture \emph{Invited review, Nature Reviews Earth and Environment.}
\item Collins, AC., Borgerhoff Mulder, A., Grote, M., {\bf Ghosh, A.}, Andrews, J., Ali, S., Hamadi, B., Khamis, H., Mzee, A., Ngwali, A., Salerno, J., Caro, T. Determining the efficacy of community-based forest management in Zanzibar. 
\item {\bf Ghosh, A.}, Engle-Stone, R., Charles, D.A., Xiuping, T., Mandel, A., Advancing household survey designs with open source geospatial tools.
\end{bibsection}

\section{Sponsored Research and Programs}
\vspace{-.1275in}
\begin{bibsection}

\item \href{gfc.ucdavis.edu}{Geospatial and Farming Systems Research Consortium (GFC)}, funded by the \href{https://www.k-state.edu/siil/index.html}{USAID Feed the Future Innovation Lab for Sustainable Intensification}; Program coordinator (PI: Robert Hijmans) \hfill {July 2015 to Sep 2019}

\item \emph{Developing Advanced Index-Insurance Products for Ghanaian Smallholder Farmers}, funded by the \href{https://basis.ucdavis.edu/}{USAID Feed the Future Innovation Lab for Assets and Market Access}; PI Robert Hijmans \hfill {May 2019 to December 2019}

\item \emph{High spatial resolution maize distribution and yield maps for Tanzania}, funded by the \href{https://basis.ucdavis.edu/}{USAID Feed the Future Innovation Lab for Assets and Market Access}; PI Robert Hijmans \hfill {May 2018 to Dec 2019}

\item \emph{Climate change impact on rice yield and food security in the riverine communities in Cambodia}, funded by the \href{http://sites.nationalacademies.org/pga/peer/index.htm}{USAID PEER}; PI Sok Serey \hfill {November 2018 - October 2020}

\end{bibsection}

\section{Technical Skills
[\href{https://github.com/ani-ghosh}{github}]
}

Computer Skills
\begin{innerlist}
    \item Programming Languages: R, Python, JavaScript
    \item Remote Sensing Image Processing: Google Earth Engine APIs, ENVI-IDL,\\ ERDAS-Imagine, PCI-Geomatica, eCognition, Pix4D
    \item Other: ArcGIS, QGIS, Fragstats, ImageJ
\end{innerlist}

\vspace{.1in} 
Software Development (R-packages)
\begin{innerlist}
    \item \href{https://cran.r-project.org/web/packages/raster/index.html}{raster : Geographic Data Analysis and Modeling} (contributor) 
    \item \href{https://github.com/rspatial/luna}{luna : satellite remote sensing tools}(co-author)
\end{innerlist}

\vspace{.1in} 
Field Equipment
\begin{innerlist}
    \item Unmanned Aerial Vehicle for near surface remote sensing applications
    \item ASD Field spectroradiometer, LiCOR Plant Canopy Analyzer
\end{innerlist}


\section{Teaching Experience}
University of California, Davis \hfill {2016, 2018}
\begin{innerlist}
\item[] Environmental Analysis using GIS, co-instructor
\item[] Quantitative Geography, co-instructor
\end{innerlist}
\vspace{.1in}
TERI University, India
\begin{innerlist}
\item[] Visiting Faculty, Applied Mathematics  \hfill {Feb -- May, 2014}
\item[] Advances in Remote Sensing \hfill {2011-2013}
\item[] Digital Image Processing and Information Extraction  \hfill{2011-2013}
\end{innerlist}

\section{Student Supervision}
Royal University of Agriculture, Cambodia
\begin{innerlist}
\item Nut Nareth, Sourn Taingaun (PhD Student) \hfill{2017-current}
\end{innerlist}

\section{Workshops/ Training}
Organizing and Teaching roles
\begin{innerlist}
    \item Using R as an Open Source Geographic Information System, 2019 Training & Symposium Arkansas GIS Users Forum, Eureka Springs, AR, October 2019. 
    \item Spatial Distribution Modeling with R, GeoVet 2019, Davis, CA, October 2019.
    \item Remote sensing for agricultural insurance, RCMRD, Nairobi, Kenya, July 2019.
    \item Spatial analysis with R, Phnom Penh, Cambodia, December 2018.
    \item 2\textsuperscript{nd} International Workshop on Advanced R \& R-QTL, ICRISAT, India, December 2018.
    \item Program committee for CGIAR-SIIL track, FOSS4G 2018, (Free and Open Source For Geospatial 2018), Tanzania, August 2018.
    \item Spatial Data Analysis and Modeling for Agricultural Development, with R. Tanzania, August 2016.
    \item Introduction to Google Earth Engine platform: Introduction and hands-on workshop, Davis, CA, April 2016.
\end{innerlist}


\section{Online training materials}
\vspace{-.1275in}
\begin{bibsection}
\item \href{https://rspatial.org/}{Spatial Data Science with R} (co-author)
\item \href{https://gfc.ucdavis.edu/events/agrin/html/}{Remote sensing for agricultural insurance} (co-author)
\end{bibsection}

\section{Professional Service}
\begin{innerlist}
\item Coordinator, Center for Spatial Sciences, University of California, Davis
\item Scientific Committee member of the Academic Track of FOSS4G 2018 (Free and Open Source For Geospatial 2018), Tanzania, August 2018
\end{innerlist}
\newpage
Journal Reviewers \hfill {2011 -- Present}
\begin{innerlist}
    \item[] \emph{Applied Geography, Applied Sciences, Arabian Journal of Geosciences, Ecological\\ Indicators, Egyptian Journal of Remote Sensing, Environmental Earth Sciences,\\ Environmental Research Letters, European Journal of Remote Sensing, Geocarto\\ International,  GIScience \& Remote Sensing, IEEE Transactions on Geoscience \\and Remote Sensing, International Journal of Applied Earth Observation and \\Geoinformation, International Journal of Image and Data Fusion, International \\Journal of Remote Sensing, ISPRS Journal of Photogrammetry and Remote \\Sensing, Journal of Applied Remote Sensing, Landscape and Urban Planning, \\PeerJ, Remote Sensing Applications: Society and Environment, Remote Sensing \\of Environment, Remote Sensing, Sensors, Tropical Ecology}
\end{innerlist}

\section{Awards}
Fellowships
\begin{innerlist}
\item Junior and Senior Research Fellowship, CSIR, Government of India\hfill 2009 -- 2014 
\item Merit Scholarship from \emph{Minist{\`e}re de l\`{}{\'E}ducation, du Loisir	et du	Sport	du Qu{\'e}bec} for Post-Doctoral Research at McGill University, Canada \emph{(not availed)}\hfill 2014
\item DAAD Exchange Fellowship for PhD Students\hfill Sep 2012 -- Nov 2012
\end{innerlist}

\halfblankline

Travel Awards
\begin{innerlist}
\item Academic Federation Faculty Research Travel Grant, UC Davis for Participating in FOSS4G Annual Conference, Dar es Salaam, Tanzania\hfill August 2018
\item Young	Scientist Travel Fellowship	Award, SERB (DST-India) for	Participating\\ in SPIE Remote Sensing conference, Dresden, Germany\hfill Sep 2013
\item SPIE Partial Travel Fund\hfill Sep 2013
\end{innerlist}

\halfblankline

Imagery/Instrument Awards
\begin{innerlist}
\item DigitalGlobe Imagery Grant\hfill 2012
\item Alexander Goetz Instrument Support Program \emph{(not availed)}\hfill 2012
\end{innerlist}

% Add a little space to nudge next ``Conference Publications'' marginpar
% down to make room for tall ``Submitted Journal Publications''
% marginpar. If there are enough submitted journal publications, this
% space will not be needed (and should be removed).
%\vspace{0.1in}

%\section{Papers in Preparation}
%\vspace{-.1in}

\section{Proceedings}
\vspace{-.1275in}
\begin{bibsection}
\item Shew, A., {\bf Ghosh, A.}, Using multi-temporal remote sensing data to understand the spatio-temporal patterns of dry season rice production in Bangladesh. \emph{ISPRS Annals of the Photogrammetry, Remote Sensing and Spatial Information Sciences, Volume IV-4/W2}, 2017.
\item {\bf Ghosh, A.}, Fassnacht, FE., Joshi, PK., Koch, B., Tree species mapping using multi-sensor (hyperspectral \& LiDAR) data: Effect of sensor, scale, and classifier. \emph{Land-use Related Biodiversity in India: Seminar Proceedings}, 2014.
\item {\bf Ghosh, A.}, Joshi, PK., Identification of bamboo patches in the lower Gangetic plains using very high resolution WorldView 2 imagery. \emph{Proc. SPIE 8893, Earth Resources and Environmental Remote Sensing/GIS Applications IV, 889310}, 2013.
\end{bibsection}

\section{Conference Presentation}
\vspace{-.1275in}
\begin{bibsection}
\item Shew, AM., {\bf Ghosh, A.}, Track, JB., Nalley, LL., Estimating Crop Planting Delays and Phenological Variation caused by Extreme Rainfall Events in the US Lower Mississippi River Basin using Earth Observation Systems. \emph{AGU Fall Meeting 2019} \hfill{Dec 2018}
\item Wang, H., {\bf Ghosh, A.}, Hijmans, RJ., Climate-Driven Changes in Rice Phenology in California. \emph{AGU Fall Meeting 2019} \hfill{Dec 2018}
\item Springborn, M.R., Weill, J.A., Lips, K.R., Ibanez, R.,{\bf Ghosh, A.}, Ecosystem Service Impacts of Ecological Disruption: Evidence from Malaria Incidence in Central America. \emph{GeoVet 2019. Novel spatio-temporal approaches in the era of Big Data, Davis, CA}. \hfill {Oct, 2019}
\item Shew, AM., Owens, P.R., {\bf Ghosh, A.}, Solving Agricultural and Environmental Problems: GIS Today and Tomorrow. \emph{AR GIS Users Forum and OPENGATE Outreach Partnership Conference, Little Rock, AR}. \hfill {Nov, 2018}
\item Wang, H., {\bf Ghosh, A.}, Mapping paddy rice cropping pattern and phenology in Cambodia. \emph{FOSS4G 2018, Dar es Salaam, Tanzania.} \hfill{Aug 2018}
\item Lwehabura, J., Stewart, ZP., Rubyogo, JC., Prasad, PVV., Mason, N., {\bf Ghosh, A.}, Geospatial Analysis to Spur Technology Adoption for Increasing Bean Productivity in Tanzania. \emph{FOSS4G 2018, Dar es Salaam, Tanzania.} \hfill{Aug 2018}
\item Tiedeman, K., {\bf Ghosh, A.}, Landcover classification with R and Google Earth Engine to predict Human-elephant conflict. \emph{FOSS4G 2018, Dar es Salaam, \\Tanzania.} \hfill{Aug 2018} 
\item Lwehabura, J., Stewart, ZP., Rubyogo, JC., Prasad, PVV., {\bf Ghosh, A.}, Mason, N., Increasing Technology Adoption and Scaling through Mother-Baby Trials Paired with Geospatial Analysis of Enabling Biophysical and Socioeconomic Conditions. \emph{ 1st International Sustainable Agricultural Intensification and Nutrition (SAIN)\\ Conference, Phnom Penh, Cambodia}.
\hfill{Jan 2018}.
\item Shew, AM., {\bf Ghosh, A.}, Durand-Morat, A., Nalley, L.L., Food security impacts from the adoption of dry season hybrid and HYV rice production in Bangladesh.\emph{3rd International Conference on Global Food Security, Cape Town, South Africa}. 
\hfill{Dec 2017}.
\item {\bf Ghosh, A.}, Mandel, A., Hijmans, RJ., Developing Scalable Information Extraction Processing Pipelines using R for Earth Observation Applications.\emph{FOSS4G 2017, Boston, USA.} \hfill{Aug 2017}
\item Mandel, A., {\bf Ghosh, A.}, Hijmans, RJ., Rspatial.org, tutorials for learning Spatial R.\emph{FOSS4G 2017, Boston, USA.} \hfill{Aug 2017}
\item Shew, A., {\bf Ghosh, A.}, Using multi-temporal remote sensing data to understand the spatio-temporal patterns of dry season rice production in Bangladesh.\emph{2\textsuperscript{nd}\\ International Symposium on Spatiotemporal Computing, Boston, USA.} \hfill{Aug 2017}
\item {\bf Ghosh, A.}, Joshi, PK., Garg, RD., 2013, Potential	of hyperspectral sensors to	identify major invasive	plant	species	in lower Himalayan foothills. \emph{ 5\textsuperscript{th} Indian Youth Science Congress, Visva-Bharati University, India.}\hfill Dec 2013
\item {\bf Ghosh, A.}, Joshi, PK., Identification of bamboo patches in the lower Gangetic plains using very high resolution WorldView 2 imagery. \emph{SPIE Remote Sensing, Dresden, Germany.}\hfill Sep 2013
\item Fassnacht, FE., Latifi, H., {\bf Ghosh, A.}, Joshi, PK., Koch, B., Assessing the potential of hyperspectral imagery to map bark beetle-induced tree mortality. \emph{SPIE Remote Sensing, Dresden, Germany.}\hfill Sep 2013
\item {\bf Ghosh, A.}, Fassnacht, FE., Joshi, PK., Koch, B., Tree species	mapping using Hyperspectral	imagery: effect	of scale and classifier. \emph{ 8\textsuperscript{th} EARSeL Imaging\\ Spectrometry Workshop, Nantes, France.}\hfill April 2013
\item Chakraborty, A., Joshi, PK., {\bf Ghosh, A.}, Areendran, G., Mapping biomes of India using Holdridge Life Zone Model -- identifying footprints of climate change. \emph{13\textsuperscript{th} ESRI India User Conference.}\hfill 2012
\item Sharma, R., Joshi, PK., {\bf Ghosh, A.}, Green and Grey Mosaics of Sustainable Urban Society. \emph{Planet Under Pressure, London.}\hfill 2012

\end{bibsection}

\section{Invited talks/ Events}
\vspace{-.1275in}
\begin{bibsection}
\item {\bf Ghosh, A.}, Remote Sensing Image Processing with Open Source Software. \emph{Sacramento GIS User Group, CA, January 24, 2019}.
\item {\bf Ghosh, A.}, Mapping Surface Water Change with Google Earth Engine. \emph{Maptime Davis, CA, April 24, 2018}.
\item Hollarsmith, J., Tiedeman, K., {\bf Ghosh, A.}, Remote Sensing of Kelp: preliminary results and challenges. \emph{Puget Sound Kelp Recovery Plan Workshop WebEx, March 20, 2018}.
\item Shew, A., Durand-Morat, A., Putman, W., Nalley, L., {\bf Ghosh, A.}, Estimating Global Food Security Impacts from Hybrid and HYV Rice Adoption in Bangladesh. \emph{International Hybrid Rice Symposium in Yogyakarta, Indonesia, February 27-March 1st, 2018}. 
\item Shew, A., {\bf Ghosh, A.}, Identifying Dry Season Rice Areas in Bangladesh at 30 m Resolution Using the Landsat Archive, Time Series Analysis, and Google Earth Engine. \emph{International Hybrid Rice Symposium in Yogyakarta, Indonesia, February 27-March 1st, 2018}.
\item {\bf Ghosh, A.}, Finding Remote Sensing Data. \emph{Maptime Davis, CA, February 26, 2018}.
\item {\bf Ghosh, A.}, Using OpenStreetMap Data. \emph{Maptime Davis, CA, December 1, 2017}.
\item {\bf Ghosh, A.} The U.S. Government's Global Food Security Research Strategy: From Upstream Research to Development Impact, \emph{Board for International Food and Agricultural Development (BIFAD) public meeting, Washington, DC September 12, 2017}. 
\item {\bf Ghosh, A.}, Mandel, A., Hijmans, RJ., Geospatial Consortium, \emph{USAID Workshop on Innovation for Data-Driven Agriculture, Boulder, CO, April 2017}.
\item {\bf Ghosh, A.}, \emph{OSGeo California Chapter Annual Meeting, Davis, CA, January 2016}.
\end{bibsection}

% \section{References}
% 
% Robert J. Hijmans
% \begin{innerlist}
% \item[] Professor, Department of Environmental Science and Policy\\
% University of California, Davis, CA, USA\\ 
% E-mail: rhijmans@ucdavis.edu \hfill{Phone: +1-530-752-6555}
% \end{innerlist}
% 
% \halfblankline
% 
% Pawan Kumar Joshi
% \begin{innerlist}
% \item[] Professor, School of Environmental Sciences\\
% Jawaharlal Nehru University, Delhi, India\\ 
% E-mail: pkjoshi@mail.jnu.ac.in \hfill{Phone: +91-011-26704323}
% \end{innerlist}
% 
% \halfblankline
% 
% Partha Sarathi Roy
% \begin{innerlist}
% \item[] NASI Senior Scientist Platinum Jubilee Fellow,\\
% System Analysis for Climate Smart Agriculture\\
% Innovation Systems for the Dry lands\\
% ICRISAT, Hyderabad, India\\
% E-mail: R.Parthsarathi@cgiar.org, psroy13@gmail.com
% \end{innerlist}
% 
% \halfblankline

\end{document}

%%%%%%%%%%%%%%%%%%%%%%%%%% End CV Document %%%%%%%%%%%%%%%%%%%%%%%%%%%%%

%----------------------------------------------------------------------%
% The following is copyright and licensing information for
% redistribution of this LaTeX source code; it also includes a liability
% statement. If this source code is not being redistributed to others,
% it may be omitted. It has no effect on the function of the above code.
%----------------------------------------------------------------------%
% Copyright (c) 2007, 2008, 2009, 2010, 2011 by Theodore P. Pavlic
%
% Unless otherwise expressly stated, this work is licensed under the
% Creative Commons Attribution-Noncommercial 3.0 United States License. To
% view a copy of this license, visit
% http://creativecommons.org/licenses/by-nc/3.0/us/ or send a letter to
% Creative Commons, 171 Second Street, Suite 300, San Francisco,
% California, 94105, USA.
%
% THE SOFTWARE IS PROVIDED "AS IS", WITHOUT WARRANTY OF ANY KIND, EXPRESS
% OR IMPLIED, INCLUDING BUT NOT LIMITED TO THE WARRANTIES OF
% MERCHANTABILITY, FITNESS FOR A PARTICULAR PURPOSE AND NONINFRINGEMENT.
% IN NO EVENT SHALL THE AUTHORS OR COPYRIGHT HOLDERS BE LIABLE FOR ANY
% CLAIM, DAMAGES OR OTHER LIABILITY, WHETHER IN AN ACTION OF CONTRACT,
% TORT OR OTHERWISE, ARISING FROM, OUT OF OR IN CONNECTION WITH THE
% SOFTWARE OR THE USE OR OTHER DEALINGS IN THE SOFTWARE.
%----------------------------------------------------------------------%

